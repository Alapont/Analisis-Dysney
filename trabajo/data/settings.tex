%%%%%%%%%%%%%%%%%%%%%%%%%%%%%%%%%%
%
% AQUI SE TIENE LO NECESARIO Y MÁS
% 
% Ambientes de cuadros
% Dibujar Funciones
% Dibujar Tableros
% Creación de código fuente
% estilos, tamaños, cabeceras y pie de página
%
%%%%%%%%%%%%%%%%%%%%%%%%%%%%%%%%%%%
%
% Aqui se incluyen un conjunto de instrucciones y paquetes
% necesarios para todo lo que se quiera escribir
% NOTA: es evidente que el creador de la plantilla NO HA ESCRITO
% NINGUN PAQUETE DE LOS DE ABAJO
%
% Ha recopilado y modificado todo lo que él considera necesario
%
%%%%%%%%%%%%%%%%%%%%%%%%%%%%%%%%%%%




\usepackage{xparse}
\usepackage{multirow}
\usepackage[margin=1in]{geometry} 
\usepackage{amsmath,amsthm,amssymb}
\usepackage{booktabs}
\usepackage{tikz}
\usepackage{mdframed}
\usepackage{emptypage}
\usepackage{esvect}
\usetikzlibrary{arrows,positioning,fit,shapes,calc}
\usepackage{framed}
\usepackage{fancyhdr} 
\usepackage[
    type={CC},
    modifier={by},
    version={4.0},
]{doclicense}
\pagestyle{fancy} 
\usepackage{chessboard}
\storechessboardstyle{5x5}{maxfield=e5}

%%%%%%%%%%%%%%%%%%%%%%%%%%%%%%%%%%%%%%%%%%%%%%%%%%
%CAMBIAR X1 Z1 Y A1 para el documento de TFG
\lhead[Género en películas Disney]{}
\rhead[]{Género en películas Disney}
\renewcommand{\headrulewidth}{0.5pt}
%\lfoot[OTEA - Curso]{David Pacios Izquierdo}
\renewcommand{\footrulewidth}{0.5pt}
%%%%%%%%%%%%%%%%%%%%%%%%%%%%%%%%%%%%%%%%%%%%%%%%%%

\usetikzlibrary{calc,angles,positioning,intersections,quotes,decorations.markings}
\usepackage{tkz-euclide}
\usetkzobj{all}
\usepackage{pgfplots}
\usepackage{graphicx}
\usepackage{subfigure}
\usepackage{float}
\usepackage{listings}
\usepackage{xcolor}
\usepackage{multicol}
\pgfplotsset{compat=1.5}
\usepackage{longtable}

%%%%%%%%%%%%%%%%%%%%% Dibujar funciones 
% https://es.sharelatex.com/learn/Pgfplots_package
% http://pgfplots.sourceforge.net/gallery.html
%%%%%%%%%%%%%%%%%%%%% Oficial documentation
% http://pgfplots.sourceforge.net/pgfplots.pdf
% http://pgfplots.sourceforge.net/pgfplotstable.pdf
%%%%%%%%%%%%%%%%%%%%%%%%%%%%%%%%%%%%%%%%%%%%%%%%%%%%%%%%%%%%%%%%%
\newcommand{\N}{\mathbb{N}}
\newcommand{\Z}{\mathbb{Z}}
\newcommand{\R}{\mathbb{R}}
\newcommand{\Q}{\mathbb{Q}}
\newcommand{\fd}{\rightarrow}
\newcommand{\cont}{\subset}
%%%%%%%%%%%%%%%%%%%%%%%%%%%%%%%%%%%%%%%%%%%%%%%%%%%%%%%%%%%%%%%%%
\lstset { %
    language=C++,
    backgroundcolor=\color{black!5}, % set backgroundcolor
    basicstyle=\footnotesize,% basic font setting
}
\lstset { %
    language=Octave,
    backgroundcolor=\color{black!5}, % set backgroundcolor
    basicstyle=\footnotesize,% basic font setting
}
\newenvironment{theorem}[2][Teorema]{\begin{framed}\begin{trivlist}
\item[\hskip \labelsep {\bfseries #1}\hskip \labelsep {\bfseries #2.}]}{\end{trivlist}\end{framed}}
\newenvironment{regla}[2][Regla]{\begin{trivlist}
\item[\hskip \labelsep {\bfseries #1}\hskip \labelsep {\bfseries #2.}]}{\end{trivlist}}
\newenvironment{axioma}[2][Axioma]{\begin{trivlist}
\item[\hskip \labelsep {\bfseries #1}\hskip \labelsep {\bfseries #2.}]}{\end{trivlist}}
\newenvironment{lemma}[2][Lemma]{\begin{trivlist}
\item[\hskip \labelsep {\bfseries #1}\hskip \labelsep {\bfseries #2.}]}{\end{trivlist}}
\newenvironment{exercise}[2][Ejercicio]{\begin{trivlist}
\item[\hskip \labelsep {\bfseries #1}\hskip \labelsep {\bfseries #2.}]}{\end{trivlist}}
\newenvironment{reflection}[2][Reflection]{\begin{trivlist}
\item[\hskip \labelsep {\bfseries #1}\hskip \labelsep {\bfseries #2.}]}{\end{trivlist}}
\newenvironment{proposition}[2][Proposicion]{\begin{trivlist}
\item[\hskip \labelsep {\bfseries #1}\hskip \labelsep {\bfseries #2.}]}{\end{trivlist}}
\newenvironment{corollary}[2][Corolario]{\begin{trivlist}
\item[\hskip \labelsep {\bfseries #1}\hskip \labelsep {\bfseries #2.}]}{\end{trivlist}}

\pgfplotsset{soldot/.style={color=blue,only marks,mark=*}} \pgfplotsset{holdot/.style={color=blue,fill=white,only marks,mark=*}}

\ExplSyntaxOn
\NewDocumentCommand{\ruffini}{mmmm}
 {% #1 = polynomial, #2 = divisor, #3 = middle row, #4 = result
  \franklin_ruffini:nnnn { #1 } { #2 } { #3 } { #4 }
 }

\seq_new:N \l_franklin_temp_seq
\tl_new:N \l_franklin_scheme_tl
\int_new:N \l_franklin_degree_int

\cs_new_protected:Npn \franklin_ruffini:nnnn #1 #2 #3 #4
 {
  % Start the first row
  \tl_set:Nn \l_franklin_scheme_tl { #2 & }
  % Split the list of coefficients
  \seq_set_split:Nnn \l_franklin_temp_seq { , } { #1 }
  % Remember the number of columns
  \int_set:Nn \l_franklin_degree_int { \seq_count:N \l_franklin_temp_seq }
  % Fill the first row
  \tl_put_right:Nx \l_franklin_scheme_tl
   { \seq_use:Nnnn \l_franklin_temp_seq { & } { & } { & } }
  % End the first row and leave two empty places in the next
  \tl_put_right:Nn \l_franklin_scheme_tl { \\ & & }
  % Split the list of coefficients and fill the second row
  \seq_set_split:Nnn \l_franklin_temp_seq { , } { #3 }
  \tl_put_right:Nx \l_franklin_scheme_tl
   { \seq_use:Nnnn \l_franklin_temp_seq { & } { & } { & } }
  % End the second row
  \tl_put_right:Nn \l_franklin_scheme_tl { \\ }
  % Compute the \cline command
  \tl_put_right:Nx \l_franklin_scheme_tl
   {
    \exp_not:N \cline { 2-\int_to_arabic:n { \l_franklin_degree_int + 1 } }
   }
  % Leave an empty place in the third row (no rule either)
  \tl_put_right:Nn \l_franklin_scheme_tl { \multicolumn{1}{r}{} & }
  % Split and fill the third row
  \seq_set_split:Nnn \l_franklin_temp_seq { , } { #4 }
  \tl_put_right:Nx \l_franklin_scheme_tl
   { \seq_use:Nnnn \l_franklin_temp_seq { & } { & } { & } }
  % Start the array (with \use:x because the array package
  % doesn't expand the argument)
  \use:x
   {
    \exp_not:n { \begin{array} } { r | *{\int_use:N \l_franklin_degree_int} { r } }
   }
  % Body of the array and finish
  \tl_use:N \l_franklin_scheme_tl
  \end{array}
 }
\ExplSyntaxOff
\usepackage{mdframed}
\newdimen\arrowsize
\pgfarrowsdeclare{squarea}{squarea}
{
  \arrowsize=0.4pt
  \advance\arrowsize by.275\pgflinewidth%
  \pgfarrowsleftextend{+-\arrowsize}
  \advance\arrowsize by.5\pgflinewidth
  \pgfarrowsrightextend{+\arrowsize}
}
{
  \arrowsize=0.4pt
  \advance\arrowsize by.275\pgflinewidth%
  \pgfsetdash{}{+0pt}
  \pgfsetroundjoin
  \pgfpathmoveto{\pgfqpoint{1\arrowsize}{4\arrowsize}}
  \pgfpathlineto{\pgfqpoint{-7\arrowsize}{4\arrowsize}}
  \pgfpathlineto{\pgfqpoint{-7\arrowsize}{-4\arrowsize}}
  \pgfpathlineto{\pgfqpoint{1\arrowsize}{-4\arrowsize}}
  \pgfpathclose
  \pgfusepathqfillstroke
}

\pgfarrowsdeclare{open squarea}{open squarea}%{{-.5bp}{8.5bp}}
{
  \arrowsize=0.4pt
  \advance\arrowsize by.275\pgflinewidth%
  \pgfarrowsleftextend{+-.5\pgflinewidth}
  \advance\arrowsize by7\arrowsize
  \advance\arrowsize by.5\pgflinewidth
  \pgfarrowsrightextend{+\arrowsize}
}
{
  \arrowsize=0.4pt
  \advance\arrowsize by.275\pgflinewidth%
  \pgfsetdash{}{+0pt}
  \pgfsetroundjoin
  \pgfpathmoveto{\pgfqpoint{8\arrowsize}{4\arrowsize}}
  \pgfpathlineto{\pgfqpoint{0\arrowsize}{4\arrowsize}}
  \pgfpathlineto{\pgfqpoint{0\arrowsize}{-4\arrowsize}}
  \pgfpathlineto{\pgfqpoint{8\arrowsize}{-4\arrowsize}}
  \pgfpathclose
  \pgfusepathqstroke
}


\makeatletter
\newdimen\tempa
\newdimen\tempb
\pgfdeclareshape{diamond in circle}{
\inheritsavedanchors[from=diamond] 
\inheritsavedanchors[from=circle] 
\inheritanchorborder[from=circle]
\inheritanchor[from=circle]{center}
\inheritanchor[from=circle]{radius}
\inheritanchor[from=circle]{north}
\inheritanchor[from=circle]{south}
\inheritanchor[from=circle]{east}
\inheritanchor[from=circle]{west}
\inheritanchor[from=circle]{anchorborder}
  \saveddimen\radius{

    \pgf@ya=.5\ht\pgfnodeparttextbox%
    \advance\pgf@ya by.5\dp\pgfnodeparttextbox%
    \pgfmathsetlength\pgf@yb{\pgfkeysvalueof{/pgf/inner ysep}}%
    \advance\pgf@ya by\pgf@yb%

    \pgf@xa=.5\wd\pgfnodeparttextbox%
    \pgfmathsetlength\pgf@xb{\pgfkeysvalueof{/pgf/inner xsep}}%
    \advance\pgf@xa by\pgf@xb%

    \pgf@process{\pgfpointnormalised{\pgfqpoint{\pgf@xa}{\pgf@ya}}}%
    \ifdim\pgf@x>\pgf@y%
        \c@pgf@counta=\pgf@x%
        \ifnum\c@pgf@counta=0\relax%
        \else%
          \divide\c@pgf@counta by 255\relax%
          \pgf@xa=16\pgf@xa\relax%
          \divide\pgf@xa by\c@pgf@counta%
          \pgf@xa=16\pgf@xa\relax%
        \fi%
      \else%
        \c@pgf@counta=\pgf@y%
        \ifnum\c@pgf@counta=0\relax%
        \else%
          \divide\c@pgf@counta by 255\relax%
          \pgf@ya=16\pgf@ya\relax%
          \divide\pgf@ya by\c@pgf@counta%
          \pgf@xa=16\pgf@ya\relax%
        \fi%
    \fi%
    \pgf@x=\pgf@xa%

    \pgfmathsetlength{\pgf@xb}{\pgfkeysvalueof{/pgf/minimum width}}%
    \pgfmathsetlength{\pgf@yb}{\pgfkeysvalueof{/pgf/minimum height}}%
    \ifdim\pgf@x<.5\pgf@xb%
        \pgf@x=.5\pgf@xb%
    \fi%
    \ifdim\pgf@x<.5\pgf@yb%
        \pgf@x=.5\pgf@yb%
    \fi%

    \pgfmathsetlength{\pgf@xb}{\pgfkeysvalueof{/pgf/outer xsep}}%
    \pgfmathsetlength{\pgf@yb}{\pgfkeysvalueof{/pgf/outer ysep}}%
    \ifdim\pgf@xb<\pgf@yb%
      \advance\pgf@x by\pgf@yb%
    \else%
      \advance\pgf@x by\pgf@xb%
    \fi%
  }
\backgroundpath{
    \tempa=\radius
    \pgfmathsetlength{\pgf@xb}{\pgfkeysvalueof{/pgf/outer xsep}}%
    \pgfmathsetlength{\pgf@yb}{\pgfkeysvalueof{/pgf/outer ysep}}%
    \ifdim\pgf@xb<\pgf@yb%
      \advance\tempa by-\pgf@yb%
    \else%
      \advance\tempa by-\pgf@xb%
    \fi%
    \pgfpathmoveto{\centerpoint\advance\pgf@x by\radius}%
    \pgfpathlineto{\centerpoint\advance\pgf@y by\radius}%
    \pgfpathlineto{\centerpoint\advance\pgf@x by-\radius}%
    \pgfpathlineto{\centerpoint\advance\pgf@y by-\radius}%
    \pgfpathclose%
  }
\behindbackgroundpath{
    \tempa=\radius%
    \pgfmathsetlength{\pgf@xb}{\pgfkeysvalueof{/pgf/outer xsep}}%
    \pgfmathsetlength{\pgf@yb}{\pgfkeysvalueof{/pgf/outer ysep}}%
    \ifdim\pgf@xb<\pgf@yb%
      \advance\tempa by-\pgf@yb%
    \else%
      \advance\tempa by-\pgf@xb%
    \fi%
    \pgfpathcircle{\centerpoint}{\tempa}%
  }
}
\makeatother

\newcommand{\parameter}[1]{$\langle\mbox{#1}\rangle$}
\usepackage{enumitem}
\setlength{\parindent}{12pt}
\renewcommand{\theenumi}{\Alph{enumi}}

\newcommand{\pascal}{\LARGE${\displaystyle \mathbf {P\!\!^{{}_{\scriptstyle A}}\!\!\!\!\!\;\;S\!_{\displaystyle C}\!A\!\!^{{}_{\scriptstyle L}}}}$}

\newcommand{\teflon}{\LARGE${\displaystyle \mathbf {T\!^{{}_{\scriptstyle E}}\!\!\!\!\;\;F\!_{\displaystyle L}\!O\!^{{}_{\scriptstyle N}}}}$}
%%%%%%%%%%%%%%%%%%%%%%%%%%%%%%%%%%%%%%% Más creaciones de Pascal

