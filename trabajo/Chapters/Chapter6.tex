\noindent %Agregar siempre que de por culo la indentación
\chapter{Trabajo futuro}
%A pesar de que acabamos de salvar a la humanidad y estamos camino de recoger el Nobel, es posible que nos hayamos dejado cosas por el camino, y quien mejor que nosotros para saber cómo se debe avanzar
A la hora de puntuar las películas nos hemos encontrado con diferentes obstáculos. Por un lado las métricas no son objetivas, y por lo tanto propensas a error. Por otro lado, algunas métricas utilizadas si son objetivas, pero no son representativas del sesgo de la película. El tiempo en pantalla no implica que una película sea mas o menos inclusiva, aunque puede haber una correlación. De esta manera, un trabajo futuro es la incorporación de nuevas métricas que permitan, gracias a la ley de los grandes números, converger hacia puntuaciones mas fiables de cada película.

Otro trabajo es, para las métricas que ya estamos midiendo, medirlas de forma mas precisa. Ya que una medición automática puede agilizar la incorporación de películas a la base de datos. Especialmente con algoritmos de reconocimiento de personajes basados en redes neuronales como los aplicados por programas de bancos de fotos, similares a los que pueden usar Facebook o Google.

Un estudio alternativo es el de considerar las diferentes métricas por separado independientemente de las demás y en conjunto a través de las diferentes películas. Pudiendo resultar en diferentes estudios en función del protagonismo de los personajes como la cantidad de antagonistas de cada sexo a lo largo de la historia.

Al igual que se puede mejorar la calidad de las métricas como ya hemos comentado en los dos puntos anteriores. Es importante también contar con una cantidad de muestras mayor. A la hora de escribir este trabajo Disney cuenta con 338 películas \cite{listaDisney}. Una lista que no incluye otras productoras asociadas y que también forman parte de la cultura popular. Incorporar cuantas mas películas fomenta una imagen mas clara de la evolución de las perspectivas de género en la cultura y el imaginario popular.

Otra parte que no medimos y que sería interesante llegar a medir es lo sexista del lenguaje. Ésto plantea sus propios problemas, especialmente por los parámetros a medir y como se miden y su contexto temporal.

Adicionalmente, a parte de incluir cuantas mas películas mejor, es importante sopesarlas según su impacto en la sociedad. Para lo cual habría que incluir nuevas métricas que capturar y medir. Ya que se debería sopesar su alcance, su visionado y tiempo desde la publicación (o reedición), para poder empezar a ver cuanto influye una u otra película en la sociedad de un momento dado. De esta manera conseguiríamos, no solo relacionar unas películas con otras, sino relacionar unas sociedades en el tiempo con otras. Lo cual implica enfrentarse con el problema de modelar cuanto afecta una película a la sociedad a lo largo del tiempo, lo cual no es un problema trivial. 
