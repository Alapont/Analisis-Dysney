\noindent %Agregar siempre que de por culo la indentación
\chapter{Conclusiones}
% Una vez aplicado nuestro trabajo, que conclusiones hemos obtenido. Siempre de la manera mas objetiva posible, esto es, lo menos subjetivo posible
%Lo he copiado de la wiki
    Por todos los motivos expuestos consideramos que Disney muestra un rango de personajes lo suficientemente amplio como para decir que no existe un sesgo de género significativo, si no que hacen uso de los arquetipos necesarios para contar la historia en la que se centran en el momento. Historias que pueden estar basadas en a tradición oral o ser originales. Con una inversión del género de los personajes los argumentos no sufrirían ninguna alteración y no se incurriría en la percepción del sesgo de Género.
    
\section{Análisis de los datos automáticos}
    Ateniendonos a los datos recopilados automáticamente. Podemos ver una media del 25\%. Bastante descentrada de un 50\% que a priori sería lo ideal. Además hay una gran concentración entorno al 15\%. Un valor muy sesgado, especialmente considerando que incluye películas muy nuevas como Wall-e e Iron man, incluso con películas con multitud de protagonistas como la serie de los vengadores o Capitán America Civil war, donde uno podría esperar que el hecho de que el protagonista hable mucho mas que otros personajes, queda diluida en muchos mas personajes que podrían haber ofrecido mucha mas diversidad.
    
\section{Análisis de los datos manuales}
    El análisis manual \textit{(medición de tiempos)} muestra una dispersión de los datos demasiado grande comparado con una media bastante centrada. Esto es síntoma de poca correlación entre la diferencia de asignación de tiempo en pantalla por sexos o por tiempos. Es decir, No podemos asegurar fiablemente que si una película es mas vieja o mas nueva, vaya a implicar necesariamente que varones aparezcan mas o menos tiempo que mujeres.
    Por otro lado todas las historias comprobadas excepto una (\textit{Hércules}) pasan el test de Bechdel. Lo cual hace de esa única ocurrencia algo poco significativo.
    Con estos datos, podemos pensar que Disney genera unas historias poco sexualizadas o en las que el género importa poco.
    Como anécdota, el varón que mas tiempo permanece en pantalla es Rompe Ralph (59 minutos) y la mujer que mas tiempo aparece es Anna de Frozen (62 miutos). Ambos con tiempos similares.
    Brave es la película que hemos tratado que mas tiempo otorga a mujeres, ya que es principalmente la historia de una madre y su hija en un entorno bastante aislado. Los varones que aparecen lo hacen como un gag cómico mas que como elementos de la historia. Ocasionalmente como lo que Hitchcok denomina un McGuffin.
\section{Posible explicación de las discrepancias entre los diferentes datos}
    Los datos del análisis manual están hechos sobre pocas películas, mientras que el análisis automático se realiza sobre un pool mas grande. Aunque no es motivo suficiente para invalidar nungún conjunto de datos, si que merece un poco de prudencia y un análisis posterior. Que hemos detallado en el Trabajo futuro.