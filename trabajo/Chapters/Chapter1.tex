\noindent %Agregar siempre que de por culo la indentación
\chapter{Estado del arte}
\pagenumbering{arabic}
\noindent
%Que hay hecho sobre el tema

%Aquí es donde mola meter referencias bibliográficas, se pueden sacar desde google scholar, pero con que me las dejéis anotadas con un comentario me vale. Y ya me encargaré yo de engancharla
Existe cada vez una mayor preocupación por unas figuras mas responsables en el imaginario popular. Aunque se están dando pasos en esa dirección, como Facua denunciando libros machistas\cite{LibroEstereotipo}. El imaginario no se renueva al ritmo de la sociedad, ya que seguimos contando las mismas historias de hace décadas e incluso siglos. Y aunque la tradición oral es fácil de cambiar para un padre o una madre, las películas de animación populares se han rodad en un margen que se expande mas de 80 años\cite{england2011gender}.

En una sociedad que se preocupa por conocer todo lo que come, lo que consume, lo que produce. También se preocupa por consumir contenidos de forma ética y de que todo el proceso se ajuste también a una serie de valores. Desde luego, la crianza, que se mira todo con mucho mas recelo\cite{linebarger2005infants}. También nos planteamos con mucho mas detalle que las fuentes que utilizamos para la cría evolucione con nosotros \cite{ross2004escape} y no tengamos que amoldarnos. 